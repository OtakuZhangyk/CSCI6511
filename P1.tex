\documentclass{article}

\usepackage{fancyhdr}
\usepackage{extramarks}
\usepackage{amsmath}
\usepackage{amsthm}
\usepackage{amsfonts}
\usepackage{tikz}
\usepackage[plain]{algorithm}
\usepackage{algpseudocode}
\usepackage{minted}     
\usepackage{graphicx}
\usepackage{listings}
\usepackage{makecell}
\usepackage{hyperref}

\usetikzlibrary{automata,positioning}

%
% Basic Document Settings
%

\topmargin=-0.45in
\evensidemargin=0in
\oddsidemargin=0in
\textwidth=6.5in
\textheight=9.0in
\headsep=0.25in

\linespread{1.1}

\pagestyle{fancy}
\lhead{\hmwkAuthorName\ \ \hmwkAuthorId}
%\chead{\hmwkClassCode (\hmwkClassInstructor): \hmwkTitleShort}
\chead{\hmwkClassCode :\ \hmwkTitleShort}
\rhead{\firstxmark}
\lfoot{\lastxmark}
\cfoot{\thepage}

\renewcommand\headrulewidth{0.4pt}
\renewcommand\footrulewidth{0.4pt}

\setlength\parindent{0pt}

%
% Create Problem Sections
%

\newcommand{\enterProblemHeader}[1]{
    \nobreak\extramarks{}{Problem \arabic{#1} continued on next page\ldots}\nobreak{}
    \nobreak\extramarks{Problem \arabic{#1} (continued)}{Problem \arabic{#1} continued on next page\ldots}\nobreak{}
}

\newcommand{\exitProblemHeader}[1]{
    \nobreak\extramarks{Problem \arabic{#1} (continued)}{Problem \arabic{#1} continued on next page\ldots}\nobreak{}
    \stepcounter{#1}
    \nobreak\extramarks{Problem \arabic{#1}}{}\nobreak{}
}

\setcounter{secnumdepth}{0}
\newcounter{partCounter}
\newcounter{homeworkProblemCounter}
\setcounter{homeworkProblemCounter}{9}
\nobreak\extramarks{Problem \arabic{homeworkProblemCounter}}{}\nobreak{}

%
% Homework Problem Environment
%
% This environment takes an optional argument. When given, it will adjust the
% problem counter. This is useful for when the problems given for your
% assignment aren't sequential. See the last 3 problems of this template for an
% example.
%
\newenvironment{homeworkProblem}[1][-1]{
    \ifnum#1>0
        \setcounter{homeworkProblemCounter}{#1}
    \fi
    \section{Problem \arabic{homeworkProblemCounter}}
    \setcounter{partCounter}{1}
    \enterProblemHeader{homeworkProblemCounter}
}{
    \exitProblemHeader{homeworkProblemCounter}
}

%
% Homework Details
%   - Title
%   - Due date
%   - Class
%   - Section/Time
%   - Instructor
%   - Author
%

\newcommand{\hmwkTitle}{Project\ \#1: Shortest Path}
\newcommand{\hmwkTitleShort}{Project\ \#1}
\newcommand{\hmwkDueDate}{Tuesday, February 14, 2023}
\newcommand{\hmwkDueTime}{11:59 PM}
\newcommand{\hmwkClass}{CSCI6511: Artificial Intelligence}
\newcommand{\hmwkClassCode}{CSCI6511}
%\newcommand{\hmwkClassTime}{Lecture 1}
\newcommand{\hmwkClassInstructor}{Professor Amrinder Arora}
\newcommand{\hmwkAuthorName}{Yikai Zhang}
\newcommand{\hmwkAuthorId}{GWid: G25867739}

%
% Title Page
%

\title{
    \vspace{2in}
    \textmd{\textbf{\hmwkClass}}\\
    \textmd{\textbf{\hmwkTitle}}\\
    \normalsize\vspace{0.1in}\small{Due\ on\ \hmwkDueDate\ at \hmwkDueTime}\\
    \vspace{0.1in}\large{\textit{\hmwkClassInstructor}}
    \vspace{3in}
}

\author{
    \textbf{\hmwkAuthorName}\\
    \textbf{\hmwkAuthorId}
}
\date{}

\renewcommand{\part}[1]{\textbf{\large Part \Alph{partCounter}}\stepcounter{partCounter}\\}

%
% Various Helper Commands
%

% Useful for algorithms
\newcommand{\alg}[1]{\textsc{\bfseries \footnotesize #1}}

% For derivatives
\newcommand{\deriv}[1]{\frac{\mathrm{d}}{\mathrm{d}x} (#1)}

% For partial derivatives
\newcommand{\pderiv}[2]{\frac{\partial}{\partial #1} (#2)}

% Integral dx
\newcommand{\dx}{\mathrm{d}x}

% Alias for the Solution section header
\newcommand{\solution}{\textbf{\large Solution}}

% Probability commands: Expectation, Variance, Covariance, Bias
\newcommand{\E}{\mathrm{E}}
\newcommand{\Var}{\mathrm{Var}}
\newcommand{\Cov}{\mathrm{Cov}}
\newcommand{\Bias}{\mathrm{Bias}}

\begin{document}

%\maketitle

%\pagebreak

%\setcounter{homeworkProblemCounter}{9}

\begin{homeworkProblem}[1]
    \textbf{1. Implementation}

    

    \textbf{2. Code}

    

    \textbf{3. Experimental Analysis}

    \textit{3.1 Program Listing}

    I use Python to implement the algorithm and ran 3 test cases found on the Internet. 

    Used heapq library to generate min heap. Defined a new class, Huffman\_Node, with frequency, key, left\_child, right\_child, and code. Override "\(<\)" so that comparison between two nodes = comparison between frequency of nodes (for min\_heap). Input data is defined as a String. huffman\_coding(data) counts symbols, generates leaf nodes, uses Greedy to generate parent nodes until reach a root node, then call the recursion function to apply huffman code top-down. 

    \begin{lstlisting}[language=Python]
import heapq

# define a tree node in huffman tree
class Huffman_Node:
    def __init__(self,freq,key=None,left_child=None,right_child=None,code=''):
        self.freq = freq
        self.key = key
        self.left_child = left_child
        self.right_child = right_child
        self.code = code
    # for Min-Heap, sort by Huffman_Node.freq
    def __lt__(self, other):
        return self.freq < other.freq

# A recursion function that apply huffman code top-down
def apply_code(node, huff_code_table, code=''):
    code += node.code
    if node.left_child:
        apply_code(node.left_child, huff_code_table, code)
        apply_code(node.right_child, huff_code_table, code)
    else:
        huff_code_table[node.key] = code
    return

# the input data is String
def huffman_coding(data):    
    char_freq_dict = {}
    # use min heap so that root is least frequent
    # use with Huffman_Node.__lt__
    min_heap = []
    huff_code_table = {}

    # count frequency of each charactor
    for i in data:
        if i in char_freq_dict:
            char_freq_dict[i] += 1
        else:
            char_freq_dict[i] = 1

    # generate leaf nodes of Huffman tree with chars
    for char,freq in char_freq_dict.items():
        node = Huffman_Node(freq, key=char)
        heapq.heappush(min_heap, node)

    # create the whole Huffman tree
    while len(min_heap) > 1:
        # fetch 2 nodes from min_heap with least freq
        node0 = heapq.heappop(min_heap)
        node1 = heapq.heappop(min_heap)
        node0.code,node1.code = '0','1'
        parent_node=Huffman_Node(node0.freq+node1.freq,
            left_child=node0,right_child=node1)
        heapq.heappush(min_heap, parent_node)

    apply_code(heapq.heappop(min_heap), huff_code_table)

    return huff_code_table
    \end{lstlisting}

    \textit{3.2 Test case \& Output Data}

    \begin{table}[ht]
        \centering
        \begin{tabular}{c | c | c}
            \text{Test case}
            & \text{Test Output}
            & \text{Correct\&Meet Requriement?}
            \\
            \hline
            \makecell[c]{'a'*5+'b'*9+'c'*12+\\'d'*13+'e'*16+'f'*45} & \makecell[c]{'f':'0','c':'100','d':'101',\\'a':'1100','b':'1101','e':'111'} & \textbf{Yes} \\
            \hline
            \text{'mississippi'} & 'i':'0','m':'100','p':'101','s':'11' & \textbf{Yes}\\
            \hline
            \makecell[c]{'abccdddeeeee'+'f'*8+\\'g'*13+'h'*21} & \makecell[c]{'h':'0','g':'10','f':'110','e':'1110',\\'d':'11110','c':'111110','a':'1111110','b': '1111111'} & \textbf{Yes}\\
        \end{tabular}
    \end{table}

    \textit{3.3 Time Complexity}

    The time complexity of the Huffman algorithm is \(O(n\log n)\). The min heap takes \(O(\log n)\) time to push a new value, and it happens \(2n-1\) time. The apply code top-down operation takes \(O(n)\) time, and we don't need to count frequency if given symbols and frequency as input. Thus, the time complexity is \(O((2n-1)\log n + n) = O(n\log n)\). 

~

    \textbf{4. Conclusions}

    My code implemented the algorithm and passed the test cases from the Internet. The Huffman code generated by my code meets the requirement a. and b., it is a proper Huffman coding program. 

\end{homeworkProblem}

\end{document}
